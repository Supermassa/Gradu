\section{Introduction}

Aim of the thesis is to find the accessibility estimates of how an individual firm benefits from a worker. Now the estimation is done in terms of how workers benefit from the firms. These different estimations could then be assessed side-by-side to see if there is any significant difference. In a larger picture the aim is to find out the impacts of transport investment on labour accessibility and how these impacts should be quantified. \\

Further more in the thesis the intention is to find out the impacts of transport investment on labour accessibility and how these impacts should be quantified. The data estimation is done in the context of Helsinki regional area to find the differences between benefits of individual workers and individual firms in terms of accessibility.  \\

Labour supply effects can be divided roughly on participation and employment effects. Participation effect means that an individuals labour force participation decision is an optimisation problem between costs of working, including commuting costs, and the wage \citep{venables2017}. Employment effect is related to people moving to higher paid jobs, thus increasing their tax revenue in the perspective of the government.

Transport investment has an impact on economic performance, the expectation is that the transport investment acts as a catalyst on private investment, creates jobs, boosts economic activity and all in all seen as a channel for growing the local or national economy \citep{venables2017}. \\

Transport is needed for the functionality of the labour market, as transport  enables a larger area for employees to find work. The purpose of transport investment is to enhance the performance of the labour sector. Transport means that there is better matches of firms and employees, which allows for more and better specialisation for employees. Specialisation in individual terms means more wage, which means that the tax revenue increases in national terms. This connection justifies transport infrastructure investments. \\

The cost-benefit-analysis (CBA) is a typically used framework in economic appraisal and transport infrastructure appraisal \citep{venables2007}. In this framework it is easy to compare static comparative equilibriums the "with" and "without" states of the world, when all else is held constant \citep{mackie}. In an ideal world, we would like to model all of the impacts on an individual and society level. In practice, the CBA framework is partly in monetary terms, partly in physical impacts and partly in descriptive terms. That is why conventional measures of welfare changes need to be applied with strong element of judgement \citep{mackie}. \\

The wider economic impacts (WEI) are harder to measure in the traditional CBA and they are usually considered to be impacts that go beyond the consumer surplus. WEIs are hard to take into account in traditional CBA as impact assessment tend to concentrate on areas where economic activity has expanded, in the expense of the areas where it has been displaced \citep{venables2017}. This displacement effect is one of the key factors, why the CBA framework is not the best fit for transport infrastructure appraisal.  The local effects and national effects are different and need to be addressed differently in the decision making. \\

Direct impacts are quantitative and rather easy to measure. Those ones contain the time savings of the improved transport, more efficient mobility and the enhancing effect of the improved transport to productivity \citep{laakso}. Easier access to the centre causes the equilibrium employment to increase in the city \citep{venables2007}. The direct impacts mean that the area where workers might live gets bigger and all in all it means that the labour force of the certain area is bigger. Additional employment in certain areas therefore also raises the productivity of existing workers, who now reap the benefits of larger urban agglomeration \citep{venables2007}. \\

The components of direct transport benefits are time related (time savings and reliability) and money related (vehicle operating costs and out of pocket costs/fares/tolls). Wider impacts are driven by accessibility changes and these changes are dominated by changes in travel time \citep{mackie}. \\


Indirect impacts are characterised as wider economic impacts or externalities based impacts. The indirect impacts are a little harder to measure, but usually the effects should be emphasised a little more in the decision making of transport investment. These impacts are agglomeration benefits and the imperfection of the markets, this imperfection allows for the wider economic impacts to emerge \citep{laakso}. Transport improvements typically increase the strength of agglomeration economies by increasing connectivity within the spatial economy \citep{melo}. \\

After the introduction, I go through the literature in steps. The aim is to give understanding of the literature and go through the impacts from the narrow to the broader ones.  Firstly, I examine the wider economic impacts and the labour market impacts in more detail, the emphasis is on the impacts on labour supply. The wider economic impacts are also referred as indirect impacts in the literature. I also build a short overview of the empirical literature and the results concerning labour supply. Secondly, I examine the cost-benefit analysis (CBA) as a framework on transport appraisal. This overview goes through the direct impacts of transport investment and the shortcomings of this framework to include wider economic impacts. Thirdly, I derive the logsums for accessibility changes, which can be used in the estimation of the data and to go through the different alternatives how we could measure these changes in the economic activity. \\

