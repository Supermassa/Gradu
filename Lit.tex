\section{Literature and contribution} 

This section is dedicated on different literature concerning the subject and contribution of the thesis. The objective of this section is go from the narrow and specific impacts to more broader impacts and the evaluation of these impacts. The general thought behind justifying transport investment is the following. Improvements in transport mean a reduction in transportation costs and travel times. These effects mean that existing firms may serve larger market areas. Firms are able to increase their total output and reap economies of scale. This in turn is spread out to other sectors and regions through a demand-driven multiplier process \citep{oosterhaven}. \\

Firstly, I go through the different urban economics model and how they influence transport infrastructure evaluation. Secondly, I go through the wider economic impacts and dwell a little deeper on the labour supply effects of transport infrastructure improvement. Also some empirical literature concerning labour supply effects is introduced. Thirdly, I provide the current framework used in transport infrastructure project evaluation (CBA). Fourthly, I will bring forth some points on how to develop current project evaluation framework and what is the current state of measurement of wider economic impacts concerning transport infrastructure improvement. \\

\subsection{Urban economics models}

The development of urban economics models from monocentric models to polycentric models. Because we are interested mostly on commuting costs and transport on individual and household level, this brief analysis of the urban models is restricted on these factors. \\

Monocentric cities expand around the central business district (CBD). The theoretical backbone of monocentric models comes from the work of \cite{alonso}, \cite{mills} and \cite{muth}. The key observation of the model is that commuting cost differences within an urban area must be balanced by differences in the price of living space. This compensating price variation means long and costly commuting trips for suburban residents \citep{brueckner}. In monocentric models, the key feature is also about the "consumption" of land, either directly or as an intermediate input in the production of housing. \\

In reality cities tend not to be monocentric, but rather to be polycentric. This means that the number of employment subcenters increase as the population and the commuting costs of cities increase \citep{mcmillen}. In the research conducted by \cite{fujita} the spatial configuration of the city is treated as an outcome of the interactions between households and firms. Firms favour concentration as a reason for agglomeration economies and households follow the employment distribution. The main parameters are commuting and production level. The agglomeration economies is attached to the model via a concept of "locational potential" developed by \cite{fujita}. Research has been conducted in relation to developing the theory of polycentric models even further as the monocentricity is not theoretically complete and in reality the urban land use pattern seems untenable \citep{fujita}. \\
\begin{figure}[H]
\centering
\includegraphics[scale=0.7]{Subcenter}
\caption{The bid rent curves of firms and households (above) and the land-use of the city with the development of transport \citep{hsl}}
\label{fig:Subcenter}
\end{figure}

Figure \ref{fig:Subcenter} above shows the different bid rent curves of the firms and households and the land-use within the city, when the transport infrastructure is improved. As the bid rents and land-use changes are slightly out of scope regarding this thesis, I will not go through these changes too deeply. In the figure it is well depicted how the transport investment affects the subcenters of the city and the area of the city.  \citep{hsl}. \\

The monocentric and polycentric models are interesting regarding this thesis, because of the models implications on commuting costs and transport. In monocentric models the commuting is seen as a mean to enter the central business district (CBD). On the other hand in the polycentric models the economic activity is more divided on subcenters, but usually firms favour concentration because of the agglomeration economies. The figure discussed above has relevance in both the monocentric and polycentric models. Transport investment that is conducted in a radial traffic route, the willingness to pay for land increases relatively more as we go farther from the CBD. The accessibility of firms in the subcenter is improved in relation to the CBD. Thus the residential sector expands, which also means that the city expands on the traffic route, where the improvement was made \citep{hsl}.  \\

\begin{figure}[H]
\centering
\includegraphics[scale=0.6]{Wider}
\caption{The effects of a transport improvement \citep{venables2017}}
\label{fig:Wider}
\end{figure} 

In the above figure it is shown how a transport improvement has a direct effect on the user-benefits through the generalised travel cost and change in journeys. User-benefits are the social values of transport \citep{venables2017}. The investment affects the costs of transport. The generalised travel costs are a key ingredient on the user-benefits. The costs of transport affect the accessibility of different areas, which in turn influence the decision of positioning and the land use of firms, individuals and the public sector. The impacts are implemented in the traditional CBA \citep{laakso}. These direct impacts have an effect on the wider economic impacts through productivity, private investment and land-use changes and labour market \citep{venables2017}. The labour market outcomes are the most interesting regarding this thesis. Transport may enable labour market participation on the labour supply side and jobs will be created in some places and some activities, and possibly lost in other. The question of supply and demand for labour are intertwined. In this thesis, the labour supply side is emphasised \citep{venables2017}. \\


\subsection{Wider economic impacts}

The wider economic impacts occur, because of transport improvement's impact on economic geography. The reasons why the transport improvement has wider economic impact is because of proximity and relocation of households and firms. The proximity and relocation affect the effective density of economic activity and productivity. These effects go beyond the direct productivity effects of faster journeys. There is strong economic interaction in dense places. This is why cities and agglomerations exist. There is substantial research that show the positive relationship between density and economic activity \citep{venables2017}.  \\

Transport investment also affects the locations attractiveness for other financial investment. User benefits are experienced by residents, workers and firms. The user benefits may activate the residential, retail and public investment. These impacts are also linked on agglomeration and productivity and have further value on the attractiveness of the location \citep{venables2017}. The improvement may also have an effect on the labour market, both in supply and demand. On the next section we go deeper on the labour market effects. \\

There are unaddressed questions on how to include the wider economic impacts on transport appraisal and should they even be included. These questions are related on market failures, the displacement effect and the quantifying and predicting of the effects \citep{venables2017}. Firstly, market failures means that the transport improvement creates additional benefits or costs. Additional meaning here benefits or costs that go over and above the user-benefits. Secondly, displacement of some work from local areas need to be taken into consideration on national aggregate level. When concentrating on national aggregate level, the view of the appraisal is more complete. Thirdly, as a single transport project has complementary effects on other transport projects, land-use and other policy changes. It is not feasible to believe that even the larger projects have transformative effects, though it is usually claimed as so \citep{venables2017}. \\

The wider economic impacts are the most interesting when thinking of accessibility of labour. These impacts are also the ones that are constantly under estimated. As \cite{laakso} say in their report to Finnish transport agency that especially the wider economic impacts are hard to take into consideration in the decision making considering transport investments. That is why an estimation of the distance function and a literature review on the accessibility of labour would have an impact on the way transport investments effects are measured in the future \citep{laakso}.\\


\subsubsection{Labour market impacts}

The transport investment may have an effect on the labour market in two different ways. Firstly, through agglomeration economies, which is made clear by the fact that productivity grows with city size or density. \citep{andersson} This effect can also be described as labour market pooling as described by \cite{marshall} as a localised industry gains a great advantage from the fact that it offers a constant market for skill. This means that labour market pooling affects the matches made by individuals and firms, making these matches more productive. Secondly, the effect of transport investment on labour market comes through the labour supply in the economy given static housing market. A reduction in travel time and thereby travel cost may increase production, since more time is used in production rather than traveling \citep{andersson}. Of these two labour market effects, the labour supply effect is the more interesting effect regarding this thesis. \\

Income taxation provides some inefficiencies in the labour market, which may have implications for evaluating transport projects. Individuals make their choices on after tax income, but the individuals' production value for society is reflected in before tax income \citep{andersson}. As \cite{venables2007} develops a model where the employees make their decisions on where to live based on the trade-off that wages are higher in the cities and the cost of commuting. Intuition behind this trade-off is that there is a tax wedge between the extra income earned by living in the city and the cost of commuting. The conclusion of the research is that the total benefit of reducing commuting costs may be several times larger than the reduction in commuting costs with some remarks and assumptions \citep{venables2007}. \\

\cite{pilegaard} formulate a spatial computable general equilibrium (SCGE) model with labour market search imperfections leading to unemployment. In this framework \cite{pilegaard} show how substantial welfare effects are omitted when evaluating the effects of a transport improvement. It should be noted that in the framework of \cite{pilegaard}, there is no income taxation and the main focus is to show how traditional CBA underestimates the impact of transport projects \citep{andersson}. \\

\subsection{Empirical literature concerning labour supply}

Most of the empirical literature considering labour supply measures the elasticity of labour supply with respect to the wage rate. The most important aspect regarding labour supply and transport investment is to examine the causal effect of travel time on labour supply \citep{andersson}.  \\

The elasticity of labour supply with respect to the wage rate has an important role in economic policy analysis. Larger elasticity of labour related to a change in tax rate corresponds to a higher excess burden of taxation \citep{evers}. The empirical estimation done by \cite{evers} is not a perfect fit to the research question concerning transport infrastructure and labour supply. All though taxation provides inefficiencies on the CBA and thus is a relevant aspect. \\

The most important aspect regarding transport infrastructure investment and labour supply is to examine the causal effect of travel time on labour supply \citep{andersson}. The causal effect might be too excessive goal to find in this thesis. More relevant is to take the first steps and assumptions related in accessibility improvement of labour supply. \\

\cite{guti} introduce a theoretical labour supply model, which is then estimated to find out the effect of commuting distance on labour supply. The authors use a worker fixed-effects approach. \\

The results indicate that there is little variation on the workdays of worker, but much more variation on the daily work hours and also the effects of commuting costs on labour supply may differ by gender \citep{guti}. This means that workers mainly increase the total labour supply through the increase on daily hours. \\

The estimated positive effect is in line with the theoretical labour model. One possible explanation for this stated effect is that workers reduce commuting costs by coming earlier to work or departing later, this is in line with bottleneck economic models \citep{guti}. In this empirical research the fundamental assumptions are that workers may choose their daily labour supply and number of workdays \citep{guti}.   \\

Same authors also measure the effects of how the commuting distance effects workers' work effort. The length of the commuting distance affects productivity through absenteeism and by workers evading responsibilities. The aim of the empirical research is to identify the effect of distance on voluntary absenteeism. There is a need for controls for subjective and objective health indicators as we are interested on voluntary absenteeism \citep{van}. \\

In \cite{andersson} the methodology used to estimate the effects of transport improvement on labour supply is presented. The additional value generated is calculated by using general elasticity of labour supply and generalized transport costs (GTC). There exists three possible problems with this approach. Firstly, the travel time savings are dependent on precision of chosen transport model. Secondly, GTC is based on value of time that is calculated from stated preference studies. In these studies it is unclear if the value of time is taken before or after taxes. Thirdly, the use of single labour supply elasticity may cause heterogeneity, which calls for an area specific elasticity of labour supply \citep{andersson} \\

\subsection{Transport investment cost-benefit analysis}

Cost-benefit analysis is useful in quantifying the effects of a project and its distributional impacts, that is why it is the primary tool for evaluating transport projects. The evaluation is done through the costs of maintaining the passage (building and maintenance costs) and the monetised net benefits \citep{liikenne}. In the cost-benefit analysis framework, the key is to find those projects that are potential pareto improvements, meaning that winners should compensate the losers to obtain unanimous consent \citep{small}. \\

The most important aspect of measuring costs and benefits is the willingness to pay, which means how much an individual is willing to pay to change his/her circumstances. In transport investment the willingness to pay can be extended to travel time savings \citep{small} The travel time savings can then be used to derive the consumer surplus of particular improvement to transportation. The aggregate benefit would be measured by new users \(Q_1 - Q_0\), who are willing to pay almost the full cost reduction (cost reduction being the reducing of waiting time for transport), lowering the full price from \(C_0\) to \(C_1\))   \(C_0 - C_1\) to those new users that are indifferent. The aggregate benefit to new users is demonstrated by the triangular area ABF in the figure below. The consumer surplus would increase from GA\(C_0\) to area GB\(C_1\) in the same figure \citep{small}. \\

\begin{figure}[H]
\centering
\includegraphics[scale= 0.7]{ConsumerSurplus}
\caption{Benefits to existing and new users \citep{small}}
\label{fig:ConsumerSurplus}
\end{figure} 
 

In the figure, the area ABF is approximately triangular. The triangle is thus equal to half of the number of new users multiplied by the reduction of full prices. This estimation is known as the rule-of-one-half (RoH). The principle greatly simplifies the estimation of benefits to new users as the estimation of the demand curve is not needed, but only the number of new users and cost savings to existing users \citep{small}. The RoH estimation has its critiques. As mentioned in \cite{geurs} when RoH is used as a practical assumption of consumer surplus, a number of assumptions are made that do not generally hold. Firstly, the assumption that the demand function is linear, which is usually true only for new infrastructure projects. Also the changes in generalised costs should be regarded as marginal if the changes in demand are large. For example, in some traffic reduction measures RoH can lead to significant errors \citep{geurs}. Secondly, the benefits of accessibility accumulating to economic agents should come from the generalised travel costs changes within the transport system with respect to rule-of-half estimation \citep{geurs}. This assumption becomes problematic when thinking of the wider economic impacts as well as the direct impacts. \\

In CBA it is hard to grasp the full picture of economic activity as the values placed on accessibility changes by individuals may differ from those of the society \citep{andersson}. One answer to this problem is as \cite{venables2017} proposes in his research about the full economic modelling exercise, where all resource constraints are properly imposed, private sector responses modelled, market imperfections are made explicit and real income (utility) benefits accurately calculated. This could be done on large enough projects, but it is not a general solution. Another solution is spatial computable general equilibrium analysis (SCGE), which relaxes the partial equilibrium assumptions by modelling the surrounding economy. It has been stated that spatial computable general equilibrium (SCGE) models offer the chance of computing the wider economic impacts theoretically in a more satisfactory way. \citep{andersson}. All in all, the market imperfections are the most interesting impacts relating to the accessibility of labour. This is why there is a need to develop a framework where the wider economic impacts of different aspects of economic activity can be quantified and applied in appraisal \citep{venables2017}. \\

In Finland, the Finnish transport agency has researched the evaluation of transport routes costs and benefits. The impacts of travel-time, accidents, noise, emissions and the operating costs of different vehicles (cars, trains, buses and so on) are analysed. These impacts are then multiplied by the designated estimated unit values. These unit values have been derived from the market or from valued prices \citep{liikenne}. This is the technical measurement of the cost-benefit analysis, which is used widely in transport infrastructure analysis. If the benefits and savings are larger than the investment, then the cost-benefit ratio is over one. Even if the cost-benefit ratio would be under one, the investment is not necessarily unprofitable. This is mainly because of the wider economic impacts. Usually in the project evaluation and cost-benefit analysis the agglomeration and labour supply accessibility impacts are left out from the evaluation \citep{goebel}. \\

The Finnish transport agency has published a report on how to improve the project evaluation from Finnish viewpoint. There are five aspects that need the most attention in project evaluation in Finland. Firstly, other than work related travel-time savings and the re-evaluation of the road usage costs. Secondly, the valuation of emissions and accidents. Thirdly, the precision impacts of railroad evaluation. Fourthly, the use of mobile devices enables the utilisation of the travel-time in a more beneficial way, which means that the travel-time savings valuation decreases. Last point is about robotisation, which enables the better use of travel-time even in passenger cars. This would mean that public transport loses one of its most important competitive advantage \citep{goebel}. The most relevant aspects for further research regarding this thesis are the first, third and fourth. \\

From this short overview, we can conclude that there is a need for a better framework in the case of transport infrastructure improvement appraisal. The cost-benefit analysis is not a sufficient tool for measuring the wider economic impacts. The wider economic impacts usually go beyond the project evaluation framework. The wider economic impacts are identified, but the evaluation of the wider economic impacts has not been instructed \citep{laakso}. \\

There has been a rise of macroeconomic literature regarding economic impacts of transport as \cite{andersson} conclude. There has been two types of approaches, either through the usage of production function or the cost function. These approaches should include the externalities better than the traditional CBA. The critique concerning production function based approaches is that the behaviour of economic agents is overlooked, the cost benefit approach is better in this regard \citep{andersson}. \\

The scope of the analysis is the main difference between the production function approach and traditional CBA. This is because of the different angles of the approaches. According to \cite{andersson} the production function is based on macroeconomic theory and on output elasticities with respect to transport infrastructure. Traditional CBA is based on microeconomic theory and on the time and cost savings and the externalities associated to transport. In traditional CBA it is hard to include wider economic impacts, in theory the case should be different in production function approach. The output elasticities estimates are spread out both in magnitude and direction. This means that the analysis is not so straightforward between these models regarding which of the approach to utilise in transport appraisal \citep{andersson}. \\

The mindset of cost function approach and traditional CBA has more similarities. As stated by \cite{andersson} in CBA the generalised transport costs (the direct costs and distance- and time dependent costs) are the source of benefits. In cost function the distance dependent costs are included as a variable in the estimation, thus creating some overlap between cost function and generalised transport costs. In theory cost function approach could be applied in the same sense as CBA to estimate the effect of infrastructure investment. If the same infrastructure project would be analysed by cost function approach and CBA. If the cost function approach would predict greater reduction in costs than CBA, the greater reduction of costs could be seen as the inclusion of wider economic impacts in the cost function approach. Main problem of this approach is the uncertainty of included elements in cost function and in generalised transport costs \citep{andersson}.  \\


\subsection{Measurement of wider economic impacts}

The research done by \cite{melo} about the meta-analysis of empirical evidence considering transport investments and economic performance suggest that investments on transport produce strong economic benefits and foster growth. The purpose of the meta-analysis is to identify sources of systematic variation in empirical findings through statistical testing of various researches about the size of the different empirical estimates. These empirical estimates of the previous literature concerning the output elasticity of transport have been affected by two main estimation issues, which are simultaneity bias and omitted variable bias \citep{melo}.  \\
 

The logsum-model would have the form that is described by the equations below. Mainly the last equation that tells the monetary value of accessibility of a particular zone for different type of people and income groups\citep{geurs}.\\

\begin{equation}
l_{piz} = log(\sum_j exp(\mu_p V_{pjiz}))
\end{equation}
In the equation $\mu_p$ is the logsum coefficient for travel purpose p. 
V the representative utility, which in a simplified mode can be determined as stated below \citep{geurs}.

\begin{equation}
V_{zijp} = \beta_p T_{zj} + \chi_{ph} ln(C_{zj} + \delta_p D_{pj} + ...)
\end{equation}

Where T is the travel time, C the travel cost and D a variable representing the attractiveness of the destination zone \citep{geurs}.

\begin{equation}
CS_{piz}^L = VoT_{ph}  \frac{1}{\beta_p}  L_{piz}
\end{equation}

First the logsums are translated into travel times by the time coefficients $\beta_p$ and next into costs by external values of time, VoT. This equation in total tells the monetary value of accessibility of zone z for a person of type i belonging to household income group h. \citep{geurs}.
